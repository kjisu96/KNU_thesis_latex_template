%%
%% TeX:UTF-8
%%
%% 경북대학교 박사 학위논문 LaTeX 양식
%%
%% @author  권지수 Jisu Kwon (mailto: kjisu96@knu.ac.kr)
%% @date    2025. 3. 7.
%%
%% @Tested in
%% Windows 11, Texlive 2024.
%%
%% @acknowledgement
%% 본 템플릿은 2가지 양식을 참고하여 작성되었습니다. 제작자 분들께 감사를 표합니다.
%% 1. 겉표지, 인준서
%%    - https://github.com/seonho/KNU-Thesis.git
%% 2. 본문
%%    - POSTECH LaTeX Thesis Template
%%    - https://github.com/lonelywing/POSTECH_thesis_template_latex.git


% @class knu-ucs.cls
%% Tested only for English, Doctor (영어, 박사논문 양식만 테스트 됨)
\documentclass[doctor,english,final]{knu-ucs}


% 추가로 필요한 패키지가 있다면 주석을 풀고 적어넣으십시오,
%\usepackage{...}
\usepackage{amsmath}
\usepackage{enumitem}
\usepackage{algorithm}
\usepackage{algpseudocode}
\usepackage{comment} 
\usepackage{physics}
\usepackage{anyfontsize}
\usepackage{lipsum}
\usepackage{geometry}
\usepackage{pmboxdraw}
% \usepackage{bbm}
% \renewcommand{\thealgorithm}{}
\newcommand{\argmax}{\operatornamewithlimits{arg\,max}}


%%
%% 문서 정보 변수 정의
%%
\title{Title of Thesis}

\author{Gildong Hong}

\major{School of Electronic and Electrical Engineering,\\
       Major in Information and Communication Engineering}

\degreedate{June \the\year}

%% Advisor
%% knu-ucs.cls 파일에서 수정해야 함


% 본문 시작
\begin{document}

%% 앞표지, 인준서
%% 일시적으로 페이지 여백 조정
\newgeometry{left=2cm,top=2cm,right=2cm,bottom=2cm}
    \maketitle
    \makeapproval
\restoregeometry

    % 목차 (Table of Contents) 생성
    \tableofcontents

    % 표목차 (List of Tables) 생성
    \listoftables

    % 그림목차 (List of Figures) 생성
    \listoffigures

    % 위의 세 종류의 목차는 한꺼번에 다음 명령으로 생성할 수도 있습니다.
    %\makecontents

%% 이하의 본문은 LaTeX 표준 클래스 report 양식에 준하여 작성하시면 됩니다.
%% 하지만 part는 사용하지 못하도록 제거하였으므로, chapter가 문서 내의
%% 최상위 분류 단위가 됩니다.
%% You cannot use 'part'

\chapter{Introduction}
\lipsum % dummy text



\section{bbb}
\lipsum


\subsection{aaa}
\lipsum


\chapter{Proposed Architecture}
It is a citation example~\cite{Moin1998}.


%% 참고문헌 시작
%% Refences
%%
\bibliographystyle{ieeetr}
\bibliography{mybib}

% 영문초록 (abstract)
\newpage
\begin{summaryenglish}
    English abstract
\end{summaryenglish}
    
%% 한글요약문 시작 (Korean summary)
\newpage
\begin{summarykorean}
    한글 요약
\end{summarykorean}

%%
%% 감사의 글 시작
%% Acknowledgement
%%
% @command acknowledgement 감사의글

% \acknowledgement[english]
    %%때로는 엄하고 때로는 부드러운 모습으로 지도 해주신 한준희 교수님, 감사드립니다.


%% 본문 끝
\end{document}
